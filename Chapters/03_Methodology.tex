\chapter{Methodology}

\section{Experimental Setup}

Details of the experimental setup.

\section{System Design}

Description of the system architecture and design.

\section{User Study: Believability Evaluation}
\label{sec:user-study-believability}

This section describes the human-subjects study we designed to test whether our approach improves the perceived believability of agent behaviour compared to the baseline architecture introduced in Generative Agents \cite{parkGenerativeAgentsInteractive2023a}. We focus the evaluation on the believability of \emph{actions} rather than only on agent personalities or prompted conversations.

\subsection{Objectives and hypotheses}
\label{subsec:objectives-hypotheses}
We evaluate two primary hypotheses:
\begin{itemize}
	\item H1 (overall believability): Participants judge agents powered by our method as more believable overall than the baseline Generative Agents architecture in matched scenarios.
	\item H2 (action believability): For the same scenario, participants flag fewer actions as ``unbelievable'' in our method than in the baseline.
\end{itemize}
We also explore two secondary outcomes: (i) perceived causal coherence of behaviour when the high-level plan is visible, and (ii) free-text reasons participants provide when they deem an action unbelievable (used for qualitative error analysis) \cite{batesRoleEmotionBelievable1994,bogdanovychWhatMakesVirtual2016,tenceAutomatableEvaluationMethod2010,xiaoHowFarAre2024}.

\subsection{Conditions}
\label{subsec:conditions}
We compare two between-system conditions on the same simulated world and character seeds:
\begin{enumerate}
	\item \textbf{Baseline (GA)}: Our faithful re-implementation of Generative Agents \cite{parkGenerativeAgentsInteractive2023a}.
	\item \textbf{Ours (Neuro-symbolic)}: The proposed system with symbolic planning and consistency checks integrated into deliberation and action selection.
\end{enumerate}
Each participant evaluates both conditions on the same character and scenario to enable within-subject comparison. Order is counterbalanced (Latin square) to reduce presentation effects.

\subsection{Participants}
\label{subsec:participants}
We target 10-15 adult participants recruited from the university community and online platforms. Inclusion criteria are English proficiency. We will run an initial pilot (3-4 participants) to validate timing and the interface, then proceed to the main study. All participants provide informed consent and can withdraw at any time without penalty.

\subsection{Materials and stimuli}
\label{subsec:materials}
The stimulus for each condition is a replay of a single random character's day in the sandbox world. To keep tasks focused on action believability, we present:
\begin{itemize}
	\item a time-lapse \emph{video replay} of the agent acting in the world; participants can control playback speed and pause/seek;
	\item an optional overlay with the \emph{high-level plan} (intentions and sub-goals) and a \emph{low-level action log}; and
	\item UI controls to mark an action as unbelievable (``thumbs down''), provide a short reason, and continue.
\end{itemize}
Replays cover the same scenario (e.g., two simulated in-game days) and use the same character profile and randomness seed across conditions, so any observed variation is attributable to the agent architecture (baseline vs.~ours) rather than scenario noise.

\subsection{Procedure}
\label{subsec:procedure}
Each session (~30 minutes) proceeds as follows:
\begin{enumerate}
	\item \textbf{Introduction.} A short scripted briefing introduces the task and the notion of believability as coherence, plausibility, and consistency within world rules \cite{bogdanovychWhatMakesVirtual2016}.
	\item \textbf{Practice.} Participants complete a 2--3 minute tutorial on the interface using a neutral example not used in the main study.
	\item \textbf{Condition A.} Watch the replay, freely scrub, and mark any actions that feel unbelievable. For each mark, add a short explanation (optional but encouraged).
	\item \textbf{Condition B.} Repeat with the other system. Order is counterbalanced.
	\item \textbf{Summary ratings.} For each condition: provide (i) an overall believability rating (7-point Likert), (ii) a perceived causal coherence rating (7-point Likert), and (iii) a preference judgment (forced-choice which was more believable and why).
\end{enumerate}
We record time-on-task and whether the plan overlay was opened, to analyse how explanations affect believability judgments.


\subsection{Measures}
\label{subsec:measures}
We operationalize believability with complementary participant-reported and behaviour-linked measures. Unless noted, higher values indicate higher believability.

\paragraph{Primary outcomes}
\begin{enumerate}
	\item \textbf{Overall believability (Likert).} Single item per condition on a 7-point scale with anchors: 1 ``not at all believable'', 4 ``moderately believable'', 7 ``extremely believable''. The item prompt is: ``How believable was the agent's behaviour overall in this replay?''
	\item \textbf{Action-level unbelievable rate (event-normalized).} Participants can flag any on-screen action as unbelievable. Let $F$ be the number of unique action events a participant flagged in a condition and $A$ the number of \emph{atomic actions} actually viewed by that participant (derived from the action log restricted to watched timestamps). The primary rate is
	\begin{equation*}
		r_{\mathrm{unbel}} = \frac{F}{A} \times 100\, ,
	\end{equation*}
	expressed as flags per 100 atomic actions. Atomic actions are the smallest logged action units (e.g., open-door, pick-up, speak, move-to). If a participant sets multiple flags within a 2-second window around the same atomic action, we merge them into one event-level flag.
	\item \textbf{Pairwise preference.} Forced-choice question: ``Which of the two replays was more believable overall?'' (Baseline vs. Ours).
\end{enumerate}

\paragraph{Secondary outcomes}
\begin{itemize}
	\item \textbf{Causal coherence (Likert).} 7-point rating of how coherent the behaviour felt as a sequence of goals and subgoals: 1 ``not coherent'', 7 ``highly coherent''.
	\item \textbf{Plan adherence (Likert).} 7-point rating of alignment between visible high-level plan and observed actions (recorded even if the plan overlay is not opened, in which case the item is skipped and treated as missing by design).
	\item \textbf{Unbelievable-action categories (coded).} Free-text reasons for each flag are open-coded into categories such as: goal inconsistency, environment rule violation, temporal implausibility, social norm violation, and low-level control failure. Two independent coders label a stratified sample (≥30% of flags); disagreements are adjudicated and inter-rater agreement (Cohen's $\kappa$) is reported.
\end{itemize}

\paragraph{Logged covariates (for analysis, not outcomes)} We log condition order, scenario ID, participant playback time, number of overlay openings, and self-reported prior experience with simulations/games. These are used as covariates in exploratory models and to check for order effects.

\subsection{Data quality and exclusion}
\label{subsec:quality}
%% TODO: double-check exclusion criteria, is that what we want?
Sessions are excluded if participants fail an attention check (simple comprehension question about the replay), leave more than half the session unanswered, or complete in less than one-third of the median time. We pre-register exclusion rules prior to data collection.

\subsection{Analysis}
\label{subsec:analysis}
%% TODO: what kind of analysis we would like to do? Give references and a small explanation of the chosen methods.
We analyse overall believability with within-subject comparisons (paired \emph{t}-test when normality holds; otherwise Wilcoxon signed-rank). For action-level data, we fit a mixed-effects logistic regression on the probability that an action is flagged as unbelievable: \texttt{flag ~ condition + (1|participant) + (1|scenario)}. We report effect sizes (Cohen's $d$ or odds ratios) and 95\% CIs. Qualitative reasons are open-coded into categories of failure (e.g., goal inconsistency, environment rule violation) to contextualize quantitative effects.

\subsection{Ethics}
\label{subsec:ethics}
%% TODO: Will we ask DTU board? 
The study involves only minimal risk. No personal data beyond demographics is collected; all logs are anonymized and stored on encrypted drives. We will seek approval from the institutional ethics board prior to recruitment.

\subsection{Power and timing}
\label{subsec:power}
%% TODO: What is that? Do we need to have it?
A conservative power analysis for a within-subject design with a moderate effect (Cohen's $d=0.5$, $\alpha=0.05$, power $=0.8$) suggests $N\approx34$. We therefore aim for 24--36 valid participants after exclusions; the pilot is analysed descriptively and may inform small interface adjustments.

%%% TODO: I don't think it is necessary to have this subsection in master thesis
\subsection{Preregistration and availability}
\label{subsec:prereg}
We will preregister hypotheses, exclusion rules, and primary/secondary outcomes, and release the anonymized dataset, analysis scripts, and the evaluation interface after publication.
