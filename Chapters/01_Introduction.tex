\chapter{Introduction}
\label{ch:introduction}

\section{Background and Context}
\label{sec:background-context}

Large Language Model (LLM)-driven generative agents can simulate complex, human-like behavior in virtual worlds, games, and social scenarios by leveraging commonsense reasoning and natural language capabilities \cite{parkGenerativeAgentsInteractive2023a}. However, purely neural approaches produce logical inconsistencies: agents attempt impossible actions (opening nonexistent doors), violate temporal constraints (simultaneous commitments), or pursue conflicting goals, undermining believability \cite{batesRoleEmotionBelievable1994}.

Park et al.\ \cite{parkGenerativeAgentsInteractive2023a} demonstrated emergent social dynamics through memory streams, reflection, and hierarchical planning. Yet their planning component lacks mechanisms to verify logical consistency or enforce environmental constraints, producing plans that are contextually plausible but violate hard constraints or exhibit temporal inconsistencies.

Neuro-symbolic AI addresses this by combining neural generation with symbolic planning. LLMs generate Planning Domain Definition Language (PDDL) schemas encoding constraints and action preconditions \cite{tantakounLLMsPlanningModelers2025}; symbolic validators then verify plans, providing guarantees about constraint satisfaction. This retains LLM flexibility while introducing structured verification to detect and repair invalid plans before execution.


\section{Problem Statement}
\label{sec:problem-statement}
% I don't thik this is a good problem statement it should follow more closely our thessis propositons but still be concise Title
% Neuro-Symbolic Planning for Enhancing Coherence and Believability in Large Language Model-Driven Agents
% Description (Problem Statement)
% Generative agents powered by Large Language Models (LLMs) can simulate complex, human-like behavior in interactive environments, but often suffer from logical inconsistencies and incoherent action sequences. For example, agents may attempt actions that violate environmental constraints, which undermines the plausibility of their behavior.
% This project investigates how to improve coherence and believability by replacing hierarchical planning with a neuro-symbolic planning framework. In particular, we will explore approaches where LLMs generate Planning Domain Definition Language (PDDL) schemas, combining symbolic planning for logical consistency with LLMs for natural language understanding and commonsense reasoning. The effectiveness of the approach will be evaluated in simulation environments, measuring both environmental constraint adherence and human-perceived believability.

Existing agent architectures face a fundamental tradeoff: purely symbolic systems ensure logical consistency but lack flexibility and commonsense reasoning, while purely neural LLM-driven agents offer adaptability but produce logically inconsistent plans.

\textbf{Research question:} How can a neuro-symbolic planning framework improve the coherence and believability of LLM-driven generative agents?


\section{Research Aim and Objectives}
\label{sec:research-aim-objectives}

\textbf{Aim:} To develop and evaluate a hybrid neuro-symbolic planning system in which an LLM generates a hierarchical plan, the plan actions and environment constraints are formalized in PDDL, and a symbolic validator verifies logical consistency, identifies constraint violations, and guides iterative plan refinement.

\textbf{Objectives:}

\begin{enumerate}
    \item Reimplement the generative agents architecture \cite{parkGenerativeAgentsInteractive2023a} with a modular, extensible codebase that supports integration of symbolic planning components.
    \item Design and implement a PDDL-based validator that formalizes environmental constraints, detects planning violations, and outputs actionable diagnostic feedback.
    \item Develop visualization and explanation tools that make agent plans, constraint violations, and repair proposals inspectable to researchers and evaluators.
    \item Evaluate the system using (a) quantitative metrics (constraint violation rates, plan success rates, and repair efficiency) comparing a baseline hierarchical planner against the neuro-symbolic approach, and (b) qualitative human evaluation of perceived believability.
\end{enumerate}


\section{Methodological Overview}
\label{sec:methodological-overview}

This study extends the generative agents architecture \cite{parkGenerativeAgentsInteractive2023a} by replacing hierarchical planning with a neuro-symbolic framework. The LLM generates hierarchical plans and PDDL action schemas; a symbolic validator detects constraint violations. Evaluation combines:

\begin{itemize}
    \item \textbf{Quantitative metrics}: Constraint violation counts and rates on matched scenarios comparing (i) baseline hierarchical planning and (ii) neuro-symbolic planning with validator-guided revision. Metrics include violations per 100 actions, plan success rates, and repair efficiency.
    \item \textbf{Qualitative assessment}: Within-subjects user study comparing perceived believability of agent behaviors from both systems.
\end{itemize}


\section{Scope and Limitations}
\label{sec:scope-limitations}

This project focuses on simulation environments with deterministic action effects and complete observability. Real-world robotics introduces sensing uncertainty and physical dynamics beyond our scope. Evaluation constraint adherence, and perceived believability rather than real-time performance or scalability to large multi-agent systems.


\section{Thesis Structure}
\label{sec:thesis-structure}

The remainder of the thesis is organized as follows:

\begin{itemize}
    \item \textbf{Chapter 2: Theoretical Background} --- Establishes core concepts (LLMs, agents, planning paradigms including PDDL) and reviews relevant literature.
    \item \textbf{Chapter 3: Methodology} --- Describes the system design, experimental setup, and evaluation protocols. Details the symbolic validator architecture and the within-subjects user study for assessing believability and constraint adherence.
    \item \textbf{Chapter 4: Results} --- Reports quantitative constraint-violation metrics and qualitative believability findings from the user study.
    \item \textbf{Chapter 5: Discussion} --- Interprets results, situates findings within the literature, and discusses limitations and implications for agent design.
    \item \textbf{Chapter 6: Conclusion and Future Work} --- Summarizes contributions and suggests directions for future research.
\end{itemize}
