\chapter{AI Tools and Configuration Details}
\label{app:ai_details}
[To be completed]

\chapter{PDDL Example: Student Scheduling Domain}
\label{app:pddl-example}

This appendix provides a complete, runnable PDDL example for the student scheduling domain introduced in \cref{ch:background}. It consists of a domain file (predicates and action schemas), a problem file (objects, initial state, and goals), and a valid plan solving the problem.

\section{Domain File}

\begin{lstlisting}[language=lisp, caption={Student Scheduling Domain}, label=lst:pddl-domain]
(define (domain student-scheduling)
  (:requirements :typing)
  
  (:types
    student location time - object
    course assignment - object
  )
  
  (:predicates
    (at ?s - student ?l - location)
    (enrolled ?s - student ?c - course)
    (assignment-completed ?s - student ?a - assignment)
    (lecture-scheduled ?c - course ?l - location ?t - time)
    (now ?t - time)
  )
  
  (:action move
    :parameters (?s - student ?from ?to - location)
    :precondition (at ?s ?from)
    :effect (and (not (at ?s ?from)) (at ?s ?to))
  )
  
  (:action attend-lecture
    :parameters (?s - student ?c - course ?l - location ?t - time)
    :precondition (and
      (at ?s ?l)
      (enrolled ?s ?c)
      (lecture-scheduled ?c ?l ?t)
      (now ?t)
    )
    :effect (and
      (not (now ?t))
    )
  )
  
  (:action complete-assignment
    :parameters (?s - student ?a - assignment)
    :precondition (at ?s ?a)
    :effect (assignment-completed ?s ?a)
  )
)
\end{lstlisting}

\section{Problem File}

\begin{lstlisting}[language=lisp, caption={Student Scheduling Problem}, label=lst:pddl-problem]
(define (problem student-day)
  (:domain student-scheduling)
  
  (:objects
    alice - student
    home lecture-hall assignment-desk cafe - location
    math history - course
    math-homework history-essay - assignment
    morning afternoon - time
  )
  
  (:init
    (at alice home)
    (enrolled alice math)
    (enrolled alice history)
    (lecture-scheduled math lecture-hall morning)
    (lecture-scheduled history lecture-hall afternoon)
    (now morning)
  )
  
  (:goal (and
    (assignment-completed alice math-homework)
    (assignment-completed alice history-essay)
  ))
)
\end{lstlisting}

\section{Valid Plan}

A valid solution to the problem above is:

\begin{lstlisting}[caption={Solution Plan}, label=lst:pddl-plan]
1. (move alice home lecture-hall)
   State after: (at alice lecture-hall), (now morning)

2. (attend-lecture alice math lecture-hall morning)
   State after: (at alice lecture-hall), (now afternoon)

3. (move alice lecture-hall assignment-desk)
   State after: (at alice assignment-desk), (now afternoon)

4. (complete-assignment alice math-homework)
   State after: (assignment-completed alice math-homework), (now afternoon)

5. (move alice assignment-desk home)
   State after: (at alice home), (now afternoon)

6. (move alice home assignment-desk)
   State after: (at alice assignment-desk), (now afternoon)

7. (complete-assignment alice history-essay)
   State after: (assignment-completed alice history-essay), (now afternoon)

Goal reached: (assignment-completed alice math-homework) \checkmark
             (assignment-completed alice history-essay) \checkmark
\end{lstlisting}

\section{Plan Validity}

Each action in the plan satisfies its preconditions from the state immediately before it:

\begin{itemize}
    \item \textbf{Step 1}: Precondition \texttt{(at alice home)} is satisfied by initial state.
    \item \textbf{Step 2}: Preconditions \texttt{(at alice lecture-hall)}, \texttt{(enrolled alice math)}, \texttt{(lecture-scheduled math lecture-hall morning)}, and \texttt{(now morning)} are all true after step 1.
    \item \textbf{Steps 3--7}: Each move and assignment action is precondition-satisfied and reachable from the previous state.
    \item \textbf{Final state}: Both goal conditions are satisfied.
\end{itemize}

This example demonstrates how formal PDDL specifications enable explicit verification of plan correctness---a property leveraged in \cref{ch:methodology} for validating LLM-generated action sequences.